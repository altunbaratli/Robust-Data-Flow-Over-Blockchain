\pagenumbering{roman}
\setcounter{page}{1}

\selecthungarian

%----------------------------------------------------------------------------
% Abstract in Hungarian
%----------------------------------------------------------------------------
\chapter*{Kivonat}\addcontentsline{toc}{chapter}{Kivonat}

A pénzügyi hálózatok méretének növekedésével a pénzügyi csalások folyamatos vizsgálata egyre nagyobb számítási költséggel jár, holott az alacsony késleltetés biztosítása és sok felhasználó egyidejű kiszolgálása alapvető fontosságúak ebben a kontextusban.
A közelmúltban több próbálkozás történt hatékony csalásdetekciós szolgáltatások készítésére, melyek az adatbázis-kezelő rendszerek új generációját, ún. gráfadatbázisokat használtak.
Ezen rendszerek újszerűsége miatt jelenleg kevés ismerettel rendelkezünk azzal kapcsolatban, hogy pontosan hogyan teljesítenek csalásdetekciós terhelési profilokon és milyen optimalizációk alkalmazhatók rajtuk.

A dolgozatban három új gráfadatbázis-kezelő rendszert hasonlítok össze különböző csalásdetekciós esettanulmányokon és különböző méretű adathalmazokon.
A relációs adatbázis-kezelőkkel összevetve a gráfadatbázisok egyszerűbben használhatók esetén bizonyos gráfminták megkeresésére.
A dolgozatban a Neo4j, TigerGraph és Oracle PGX gráfadatbázis rendszereket használtam a kísérletekben szereplő adathalmazok feldolgozására és lekérdezésére.
A munka fókuszában három csalási séma azonosítása állt:
"smurfing" tranzakciók,
mobilbankban végzett pénzmosás jellegű műveletek és
az e-kereskedelmi platformokon (mint az Amazon és az eBay) jelenlévő eladók számára túlzottan pozitív értékelések hamisítása.

Mivel az adathalmazok hozzáférhetőségét a pénzügyi szervezetek bizalmas adatként kezelik, a teljesítménymérésekhez szintetikus adatgenerátort készítettem.
A nagyobb generált adathalmaz 500 ezer csomópontból és 1,2 millió élből áll.
A csalásdetekciós problémát gráflekérdezésekként fogalmaztam meg, majd elvégeztem a teljesítménykiértékelést a három adatbázis-kezelő rendszeren. A mérések alapján az implementációk már a relatív kis méretű adathalmaznál is komoly skálázhatósági problémákat mutattak.
A probléma megoldására egy dinamikus csalásdetekciós rendszert terveztem inkrementális nézetkarbantartási technikák alkalmazásával.
A dinamikus lekérdezéseket a Neo4j rendszer Cypher lekérdezőnyelvén valósítottam meg triggerek segítségével. A mérések alapján ez a dinamikus megközelítés akár 180-szoros javulást is hozhat egyes lekérdezések futásidejében.



\vfill
\selectenglish

%----------------------------------------------------------------------------
% Abstract in English
%----------------------------------------------------------------------------
\chapter*{Abstract}\addcontentsline{toc}{chapter}{Abstract}

As financial networks grow ever more extensive, the cost of fraudulent transaction inspection becomes computationally expensive.
Maintaining low latency and serving a large number of customers is essential in this context.
Recent attempts to provide efficient fraud detection services have employed a new generation of database management systems called graph databases.
However, due to the novelty of such systems, little is known about how they perform on such workloads and what type of optimizations they lend themselves to.

In this thesis, we worked with three novel graph database management systems to compare them in different fraud scenarios and dataset scales.
In comparison with traditional relational database systems, graph databases are easier to use for finding particular graph pattern matches.
We used the Neo4j, TigerGraph, and Oracle PGX graph database engines to process and query datasets in our experiments.
Our focus financial fraud scenarios include
smurfing transactions,
mobile banking money laundering, and
biased reviews in e-commerce platforms such as Amazon or eBay.
Due to the challenges of financial institutions' confidentiality policies, we developed a synthetic data generator. The generated dataset consists of up 500k vertex and 1.2M edge entries.

We formulated the fraud detection problems as graph queries and benchmarked them on the three database management systems. Even at this relatively small scale, the implementations exhibited significant scalability limits.
To mitigate this problem, we designed dynamic fraud detection queries by adopting incremental view maintenance techniques.
We formulated the dynamic fraud detection queries in Neo4j's Cypher query language and implemented a solution with triggers.
We observed promising performance gains up to 180 times during the experiments.

\vfill
\selectthesislanguage

\newcounter{romanPage}
\setcounter{romanPage}{\value{page}}
\stepcounter{romanPage}